\documentclass[a4paper, 11pt]{article}

% ---------- Packages ----------

\usepackage[utf8]{inputenc}
\usepackage[T1]{fontenc}
\usepackage[francais]{babel}
\usepackage{color}
\usepackage{tcolorbox}
\usepackage{listings}
\usepackage{changepage}
\usepackage{geometry}
\usepackage{url}
\usepackage{hyperref}
\usepackage{array}

% ---------- Congigurations ----------

\geometry{
    a4paper,
    left=23mm,
    right=23mm,
    top=20mm,
    bottom=20mm,
}

\newcommand{\commande}[1] {
    \begin{quote}
    \tt\raggedright #1
    \end{quote}
}

% ---------- Metadata ----------

\title{Fiches techniques réseau}
\author{
	Corentin Badot-Bertrand\\
	c.badot-bertrand@univ-lille1.fr
}
\date{Version du \today}

% ---------- Content ----------

\begin{document}

\maketitle

Ce document est à la fois une synthèse des différentes manipulations vues en cours de réseaux pratique, une référence pour certaines notions essentielles concernant les réseaux TCP/IP et un guide pour certaines configuration avancées.

\tableofcontents

\newpage

\section{Configuration d'un équipement}

Pour configurer un nouvel équipement réseau : 

\begin{itemize}
\item Prendre cable série spécifique au constructeur
\item Brancher prise série DB9 \textit{workstation} au \textit{patch panel}
\item Brancher RJ45 depuis \textit{patch panel} vers \textit{switch}
\\
\end{itemize}

Après configuration matériel, lancer dans un terminal

\commande{\$ minicom}

Vérifier différents paramètres (vitesse, parité, etc.) et valider avec \textit{enter} pour se connecter.

\section{Bases essentielles IOS}

Modes de configuration IOS :

\begin{itemize}
\item Mode \textit{user} pour consulter configuration
\item Mode \textit{privileged} pour administration
\\
\end{itemize}

Changer de mode de configuration dans le CLI

\commande{> enable}
\commande{\# disable}

Configuration globale équipement

\commande{\# configure terminal}

Configuration interface réseau spécifique

\commande{\# (config) interface ethernet 0/x}

Obtenir informations sur le \textit{switch} et sa configuration réseau

\commande{\# show running-config}
\commande{\# show startup-config}
\commande{\# show version}
\commande{\# show line}
\commande{\# show interface}

Complétion commande et affichage de l'aide IOS

\commande{\# comman?}
\commande{\# commande ?}

\section{Mise en place réseau VLAN}

Rappels sur \textit{Virtual Local Area Network} : 

\begin{itemize}
\item Réseau virtuel au niveau de la couche OSI 2 (\textit{datalink})
\item Pour assigner IP à équipements travaillant dans la couche OSI 2 (\textit{switch})
\\
\end{itemize}

Prendre configuration réseau ci-dessous comme exemple : 
\\

\begin{tabular}{|l|l|l|}
  \hline
  Composant & Adresse IP & Masque réseau \\
  \hline
  Réseau VLAN & 192.168.0.0 & 255.255.255.0 \\
  Workstation & 192.168.0.5 & x \\
  Switch & 192.168.0.4 & x \\
  \hline
\end{tabular}

\bigbreak
Changer adresse carte réseau \textit{workstation}

\commande{\$ ifconfig eth0 192.168.0.5 255.255.255.0}

Avec liaison série, accéder à configuration terminal

\commande{\# configure terminal}

Et définir les détails de l'interface \textit{vlan1}

\commande{(config) interface vlan1}
\commande{(config-if) ip address 192.168.0.4 255.255.255.0}

\section{Configuration protocole telnet}

Rappels à propos du protocole telnet : 

\begin{itemize}
\item Protocole pour communication avec serveur distant
\item Texte circule en clair et non-encrypté
\item Connection sur le port 23
\\
\end{itemize}

Après avoir configuré VLAN, accéder à configuration

\commande{\# configure terminal}

Définir informations d'identification pour liaison série virtuelle

\commande{(config) line vty 0}
\commande{(config-line) password XXX}
\commande{(config-line) login}

Configurer privilège pour pouvoir accéder au mode administrateur (interdit par défaut)

\commande{(config) line vty 0}
\commande{(config-line) privilege level 15}

\section{Configuration protocole SSH}

Rappels à propos du protocole SSH (\textit{Secure Shell}) : 

\begin{itemize}
\item Protocole pour communication avec serveur distant
\item Toute communication encryptée
\item Connection sur le port 22
\\
\end{itemize}

Après avoir configuré VLAN, accéder à configuration

\commande{\# configure terminal}

Configuration identification sur \textit{switch}

\commande{(config) aaa new-model}
\commande{(config) username john password 0 bonjour}

Mise en place chiffrement et protocole SSH

\commande{(config) crypto key generate rsa}
\commande{(config) ip ssh time-out 60}
\commande{(config) ip ssh authentication-retries 2}

Tester SSH depuis \textit{workstation} avec utilisateur "john"

\commande{\$ ssh john@192.168.0.4} 

Désactiver protocole telnet pour accéder au \textit{switch}

\commande{(config) line vty 0}
\commande{(config-line) transport input ssh}

\section{Backup de la configuration IOS}

En faisant tourner serveur TFTP sur \textit{workstation}, utiliser commande

\commande{\# copy running-config tftp:}

Remplir informations serveur TFTP pour obtenir copie configuration actuelle.

\section{Processus de commutation}

Quand machine \textit{ping} autre machine on observe que table de commutation est bien dynamiquement alimentée.
\\

Modifier le cache ARP avec commande dans terminal

\commande{\$ arp 192.168.10.99 12:34:56:78:9A:BC}

Modifier délai "vidange" table commutation sur \textit{switch}

\commande{\# configure terminal}
\commande{\# mac-address-table aging-time 10}

Pour afficher le nouveau délai de la table

\commande{# show mac-address-table aging-time}

Conséquences du changement de temps :

\begin{itemize}
\item Si on met un temps trop court on envoie des paquets à tout le monde  y compris aux machines à qui le paquet n’est pas destiné le paquet (risque de surcharge du réseau)
\item Si on met un temps trop long on risque d’avoir encore en mémoire trop de machines  (risque de surcharge de la mémoire du switch)
\\
\end{itemize}

Changer adresse MAC pour effectuer tests depuis \textit{workstation} 
    
\commande{\$ sudo ifconfig enp2s0 hw eth 12:34:56:78:9A:BC}

\section{Sécurisation des ports}

Pour éviter saturation de la table de commutation (plus de nouvelles adresses, mémoire saturée, etc.)
\\

Sécuriser port 1 du \textit{switch} avec la commande

\commande{(config) interface Fastetethernet 0/}
\commande{(config if) port security max 2}

Quelques observations après manipulation :

\begin{itemize}
\item Adresse A connectée est maintenant notée en \textit{secure} dans table de commutation.
\item Si adresse A est sur autre port, plus possible de communiquer avec.
\item Si plus de N connections sur port sécurisé, erreur de violation de sécurité est lancée.
\end{itemize}

\section{Boucles de commutation}

\section{Bridge Linux}

Pour transformer \textit{workstation} en \textit{bridge} Linux

\commande{\$ sudo brctl addbr nom-du-bridge}

Si nécessaire, assigner adresse IP au \textit{bridge}

\commande{\$ ifconfig nom-du-bridge 192.168.0.7}

\section{Protocole spanning-tree}

Désactiver le protocole \textit{spanning-tree} avec 

\commande{(config) no spanning-tree vlan 1}

\section{Filtres Wireshark}

Après avoir capturé réseau avec Wireshark ou en ouvrant fichier de capture, utiliser filtres pour trier paquets efficacement. Les exemples ci-dessous ont été utilisés pendant les cours réseau.
\\

Filtrer tous les paquets telnet provenant de 192.168.0.5

\commande{telnet \&\& ip.src == 192.168.0.5}

\end{document}